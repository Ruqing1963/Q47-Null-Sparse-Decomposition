\documentclass[11pt, a4paper]{article}
\usepackage[utf8]{inputenc}
\usepackage[T1]{fontenc}
\usepackage{amsmath, amssymb, amsthm}
\usepackage{geometry}
\usepackage{cite}
\usepackage{booktabs}
\usepackage{url}
\usepackage[hidelinks]{hyperref}

% Page layout
\geometry{left=3cm, right=3cm, top=3cm, bottom=3cm}

% Theorem environments
\theoremstyle{plain}
\newtheorem{theorem}{Theorem}[section]
\newtheorem{proposition}[theorem]{Proposition}
\newtheorem{lemma}[theorem]{Lemma}
\newtheorem{corollary}[theorem]{Corollary}

\theoremstyle{definition}
\newtheorem{definition}[theorem]{Definition}
\newtheorem{remark}[theorem]{Remark}
\newtheorem{assumption}[theorem]{Assumption}

% Title
\title{\textbf{On the Distribution of the Cyclotomic Norm Form\\[4pt]
$n^{47}-(n-1)^{47}$ in Arithmetic Progressions}}

\author{\textbf{Ruqing Chen}\\
GUT Geoservice Inc., Montreal, Quebec\\
\texttt{ruqing@hotmail.com}}

\date{February 2026}

\begin{document}

\maketitle

\begin{abstract}
We study the polynomial $Q(n)=n^{47}-(n-1)^{47}$, which arises as a
norm form from the cyclotomic field $\mathbb{Q}(\zeta_{47})$.
Exploiting this algebraic structure, we give a self-contained
elementary proof that for every prime~$p$ with $p\neq 47$ and
$p\not\equiv 1\pmod{47}$, the congruence $Q(n)\equiv 0\pmod{p}$ has
no solutions, and that $Q(n)\equiv 1\pmod{47}$ for all~$n$.
For primes $p\equiv 1\pmod{47}$, we show that there are exactly $46$
solutions modulo~$p$.

We define a sparse set of \emph{effective moduli}
$\mathcal{Q}_{\mathrm{eff}}$, consisting of integers whose prime
factors all satisfy $p\equiv 1\pmod{47}$, and prove that the local
density $\rho(q)$ vanishes for every
$q\notin\mathcal{Q}_{\mathrm{eff}}$.
This yields a \emph{null--sparse decomposition} of the
Bombieri--Vinogradov error sum: the error over all moduli $q\le D$
splits into a \emph{null part} (moduli outside
$\mathcal{Q}_{\mathrm{eff}}$, where the main term vanishes) and a
\emph{sparse part} (moduli in $\mathcal{Q}_{\mathrm{eff}}$, which
carries all nontrivial analytic weight).

We show that
$\#\{q\le D:q\in\mathcal{Q}_{\mathrm{eff}}\}
\asymp D(\log D)^{-45/46}$.
A Cauchy--Schwarz argument combined with a global
Barban--Davenport--Halberstam hypothesis yields a bound on the sparse
part that falls marginally short of the classical level of
distribution $\theta=1/2$.
However, under a \emph{restricted} variance hypothesis---asserting
that the average of~$E(x,q)^2$ over effective moduli has typical
size---the same method yields $\theta=1/2$ with a power saving of
$(\log x)^{-11/23}$, thanks to a double sparsity factor.
We discuss how the structural restriction to
$\mathcal{Q}_{\mathrm{eff}}$ may further facilitate bilinear-form
approaches to a level of distribution beyond~$1/2$.
\end{abstract}

\medskip
\noindent\textbf{MSC 2020:} 11N32, 11R18, 11N35

\noindent\textbf{Keywords:} cyclotomic norm form, Bombieri--Vinogradov theorem,
effective moduli, null--sparse decomposition, Barban--Davenport--Halberstam,
level of distribution, polynomial primes

%% ================================================================
\section{Introduction}\label{sec:intro}

The Bombieri--Vinogradov theorem~\cite{Bombieri1965, Vinogradov1965}
provides an averaged form of the Generalized Riemann Hypothesis for
primes in arithmetic progressions and is a cornerstone of modern
analytic number theory.
Given a polynomial $f$ of degree~$d\ge 2$, a natural question is
whether the sequence $\{f(n)\}$ admits a \emph{level of distribution}
$\theta$ in the sense that for every $A>0$,
\begin{equation}\label{eq:BVdef}
  \sum_{q\le x^{\theta}}
  \max_{\gcd(a,q)=1}
  \biggl|
    \sum_{\substack{n\le x\\ f(n)\equiv a\pmod{q}}} 1
    \;-\;
    \frac{\rho(q)}{q}\,x
  \biggr|
  \;\ll_A\;
  x(\log x)^{-A},
\end{equation}
where
\begin{equation}\label{eq:rhoq}
  \rho(q)=\#\bigl\{n\bmod q : f(n)\equiv 0\pmod{q}\bigr\}
\end{equation}
is the number of roots of~$f$ modulo~$q$.
The classical Bombieri--Vinogradov theorem yields $\theta=1/2$ for
generic sequences.
Significant improvements have been achieved in specific settings by
Bombieri, Friedlander, and Iwaniec~\cite{BFI1986}, who used
bilinear-form methods to go beyond this threshold, with further
developments by Maynard~\cite{Maynard} and others.

In this paper we study the degree-$46$ polynomial
\begin{equation}\label{eq:Qdef}
  Q(n)=n^{47}-(n-1)^{47}.
\end{equation}
The Bateman--Horn conjecture~\cite{BH1962} predicts that $Q(n)$
takes prime values with positive density (after accounting for local
obstructions), so understanding its distribution in arithmetic
progressions is a natural question.
Since $47$ is prime, the factorization
$a^{47}-b^{47}=(a-b)\prod_{k=1}^{46}(a-\zeta_{47}^k\,b)$ yields
\begin{equation}\label{eq:norm}
  Q(n)
  =\prod_{k=1}^{46}\bigl(n-\zeta_{47}^k(n-1)\bigr)
  =\mathrm{N}_{\mathbb{Q}(\zeta_{47})/\mathbb{Q}}
     \!\bigl(n-\zeta_{47}(n-1)\bigr),
\end{equation}
so $Q(n)$ is a norm form from the cyclotomic field
$\mathbb{Q}(\zeta_{47})$.
This rigid algebraic structure leads to strong local restrictions
that we now develop in detail.

%% ================================================================
\section{Local Structure and Root Counts}\label{sec:local}

The norm-form identity~\eqref{eq:norm} imposes severe constraints on
which primes can divide~$Q(n)$.
We collect these in a single proposition.

\begin{proposition}[Local root structure]\label{prop:local}
Let $p$ be a prime and write
$\omega(p)=\#\{n\bmod p:Q(n)\equiv 0\pmod{p}\}$.
\begin{enumerate}
\item[\textup{(a)}]
If $p=47$, then $Q(n)\equiv 1\pmod{47}$ for all
$n\in\mathbb{Z}$, so $\omega(47)=0$.
\item[\textup{(b)}]
If $p\neq 47$ and $p\not\equiv 1\pmod{47}$, then $\omega(p)=0$.
\item[\textup{(c)}]
If $p\equiv 1\pmod{47}$, then $\omega(p)=46$.
\end{enumerate}
\end{proposition}

\begin{proof}
\textbf{Part~(a).}
By Fermat's little theorem, $a^{47}\equiv a\pmod{47}$ for all
$a\in\mathbb{Z}$.
Therefore
$Q(n)=n^{47}-(n-1)^{47}\equiv n-(n-1)=1\pmod{47}$,
so $47\nmid Q(n)$ for any~$n$.

\medskip\noindent
\textbf{Part~(b).}
Suppose $p\neq 47$ and $Q(n)\equiv 0\pmod{p}$, i.e.\
$n^{47}\equiv(n-1)^{47}\pmod{p}$.
Since $\gcd(n,n-1)=1$, at most one of $n,n-1$ is divisible by~$p$.
If $p\mid n$, then $(n-1)^{47}\equiv 0\pmod{p}$, whence
$p\mid(n-1)$, contradicting $\gcd(n,n-1)=1$.
The case $p\mid(n-1)$ is analogous.
Therefore both $n$ and $n-1$ are units modulo~$p$, and setting
$t\equiv n(n-1)^{-1}\pmod{p}$ gives $t^{47}\equiv 1\pmod{p}$
with $t\not\equiv 1$ (since $n\not\equiv n-1\pmod{p}$).
But when $p\not\equiv 1\pmod{47}$ we have $\gcd(47,p-1)=1$, so
$x\mapsto x^{47}$ is a bijection on
$(\mathbb{Z}/p\mathbb{Z})^{\times}$, and $t^{47}\equiv 1$ forces
$t\equiv 1$, a contradiction.

\medskip\noindent
\textbf{Part~(c).}
When $p\equiv 1\pmod{47}$, the group
$(\mathbb{Z}/p\mathbb{Z})^{\times}$ has order $p-1$ divisible
by~$47$, so $t^{47}\equiv 1\pmod{p}$ has exactly~$47$ solutions.
Excluding $t\equiv 1$ leaves $46$ values of $t=n(n-1)^{-1}$, each
determining a unique $n\equiv t(t-1)^{-1}\pmod{p}$ (since $t\neq 1$
ensures $t-1$ is invertible).
Hence $\omega(p)=46$.
\end{proof}

\begin{remark}\label{rem:norm_proof}
Parts~(a)--(b) can also be deduced from the norm-form
identity~\eqref{eq:norm} and the splitting behavior of primes in
$\mathbb{Q}(\zeta_{47})/\mathbb{Q}$: a prime $p\neq 47$ divides a
norm from $\mathbb{Z}[\zeta_{47}]$ only if $p$ splits completely,
which requires $p\equiv 1\pmod{47}$.
The elementary proof above makes the paper self-contained and
clarifies the root count in case~(c).
\end{remark}

%% ================================================================
\section{Effective Moduli}\label{sec:eff}

\begin{definition}[Effective moduli]\label{def:Qeff}
We define the set of \emph{effective primes}
$\mathcal{P}_{\mathrm{eff}}=\{p\text{ prime}:p\equiv 1\pmod{47}\}$
and the set of \emph{effective moduli}
\[
  \mathcal{Q}_{\mathrm{eff}}
  =\bigl\{q\in\mathbb{N}:
    p\mid q \Longrightarrow
    p\in\mathcal{P}_{\mathrm{eff}}\bigr\}
  \cup\{1\}.
\]
In particular, $\gcd(q,47)=1$ for every
$q\in\mathcal{Q}_{\mathrm{eff}}$, since
$47\not\equiv 1\pmod{47}$.
\end{definition}

\begin{proposition}[Vanishing of local density]\label{prop:vanish}
For any $q\notin\mathcal{Q}_{\mathrm{eff}}$, the root count
$\rho(q)=\#\{n\bmod q:Q(n)\equiv 0\pmod{q}\}$ satisfies
$\rho(q)=0$.
\end{proposition}

\begin{proof}
If $q\notin\mathcal{Q}_{\mathrm{eff}}$, then $q$ has at least one
prime factor $p\notin\mathcal{P}_{\mathrm{eff}}$.
By Proposition~\ref{prop:local}\,(a)--(b), $\omega(p)=0$ regardless
of whether $p=47$ or $p\not\equiv 1\pmod{47}$.
By the Chinese remainder theorem, $\rho(q)=0$.
\end{proof}

\begin{remark}[Counting effective moduli]\label{rem:count}
The set $\mathcal{Q}_{\mathrm{eff}}$ consists of positive integers
all of whose prime factors lie in the arithmetic progression
$1\bmod 47$.
By Dirichlet's theorem, these primes have natural density
$1/\varphi(47)=1/46$ among all primes.
A classical result on integers composed of primes from a set of
relative density~$\delta$
(see, e.g., Iwaniec--Kowalski~\cite[Theorem~7.18]{IK2004}) gives
\begin{equation}\label{eq:Qeff_count}
  N_{\mathrm{eff}}(D)
  \;:=\;
  \#\{q\le D: q\in\mathcal{Q}_{\mathrm{eff}}\}
  \;\asymp\;
  \frac{D}{(\log D)^{1-1/46}}
  \;=\;
  \frac{D}{(\log D)^{45/46}},
\end{equation}
so $\mathcal{Q}_{\mathrm{eff}}$ is genuinely sparse: its counting
function grows like $D/(\log D)^{45/46}$ compared to the full
range~$\{1,\ldots,D\}$.
\end{remark}

%% ================================================================
\section{Null--Sparse Decomposition}\label{sec:decomp}

\begin{theorem}[Null--Sparse Decomposition]\label{thm:decomp}
Let $Q(n)=n^{47}-(n-1)^{47}$ and define
\[
  E(x,q)=\max_{\gcd(a,q)=1}
  \biggl|
    \sum_{\substack{n\le x\\ Q(n)\equiv a\pmod{q}}} 1
    \;-\;
    \frac{\rho(q)}{q}\,x
  \biggr|.
\]
Then for any $D\ge 1$,
\begin{equation}\label{eq:decomp}
  \sum_{q\le D}E(x,q)
  \;=\;
  \underbrace{
    \sum_{\substack{q\le D\\
    q\notin\mathcal{Q}_{\mathrm{eff}}}}
    E(x,q)
  }_{\textup{null part}}
  \;+\;
  \underbrace{
    \sum_{\substack{q\le D\\
    q\in\mathcal{Q}_{\mathrm{eff}}}}
    E(x,q)
  }_{\textup{sparse part}}.
\end{equation}
For $q\notin\mathcal{Q}_{\mathrm{eff}}$, we have $\rho(q)=0$ by
Proposition~\textup{\ref{prop:vanish}}, so the expected main term
vanishes and the null part reduces to bounding equidistribution
errors among residues that $Q(n)$ never hits modulo~$q$.
All nontrivial analytic contributions to the error sum are supported
on $\mathcal{Q}_{\mathrm{eff}}$.
\end{theorem}

%% ================================================================
\section{Quantitative Bounds via Cauchy--Schwarz}\label{sec:cond}

We now examine what can be deduced about the sparse part of the
decomposition~\eqref{eq:decomp} from variance hypotheses.

\subsection{Global variance hypothesis}\label{ssec:global}

\begin{assumption}[Global variance hypothesis]\label{ass:global}
There exists a fixed constant $B>0$ such that
\begin{equation}\label{eq:BDH_global}
  \sum_{q\le D}E(x,q)^2
  \;\ll\;
  D\,x\,(\log x)^B.
\end{equation}
\end{assumption}

\begin{proposition}[Cauchy--Schwarz bound, global
version]\label{prop:CS_global}
Under Assumption~\textup{\ref{ass:global}}, for $D\le x^{1/2}$,
\begin{equation}\label{eq:CS_global}
  \sum_{\substack{q\le D\\
  q\in\mathcal{Q}_{\mathrm{eff}}}}
  E(x,q)
  \;\ll\;
  x\,(\log x)^{(46B-45)/92}.
\end{equation}
\end{proposition}

\begin{proof}
By the Cauchy--Schwarz inequality,
\begin{equation}\label{eq:CS_step1}
  \biggl(
    \sum_{\substack{q\le D\\
    q\in\mathcal{Q}_{\mathrm{eff}}}}
    E(x,q)
  \biggr)^{\!2}
  \;\le\;
  N_{\mathrm{eff}}(D)
  \;\cdot\;
  \sum_{q\le D}E(x,q)^2.
\end{equation}
Here the first factor counts over
$\mathcal{Q}_{\mathrm{eff}}$ while the second sums over
\emph{all} $q\le D$.
Substituting~\eqref{eq:Qeff_count} and
Assumption~\ref{ass:global}:
\[
  \text{RHS of \eqref{eq:CS_step1}}
  \;\ll\;
  \frac{D}{(\log D)^{45/46}}
  \cdot D\,x\,(\log x)^B
  \;=\;
  \frac{D^2\,x\,(\log x)^B}{(\log D)^{45/46}}.
\]
For $D\le x^{1/2}$ we have $D^2\le x$ and $\log D\asymp\log x$,
so taking square roots:
\[
  \sum_{\substack{q\le D\\
  q\in\mathcal{Q}_{\mathrm{eff}}}}
  E(x,q)
  \;\ll\;
  x\cdot\frac{(\log x)^{B/2}}{(\log x)^{45/92}}
  \;=\;
  x\,(\log x)^{(46B-45)/92}.\qedhere
\]
\end{proof}

\begin{remark}[Interpretation]\label{rem:global_exp}
The exponent $(46B-45)/92$ is negative if and only if
$B<45/46\approx 0.978$.
For the standard Barban--Davenport--Halberstam value $B=1$, the
exponent is $\tfrac{46-45}{92}=\tfrac{1}{92}>0$, so the bound
becomes $x(\log x)^{1/92}$, which does not yield the arbitrary
savings $x(\log x)^{-A}$ required for $\theta=1/2$ in the sense
of~\eqref{eq:BVdef}.
Thus the Cauchy--Schwarz method with a global BDH hypothesis falls
marginally short: only a single power of $N_{\mathrm{eff}}(D)$
enters the bound, and it nearly but not quite compensates the
variance.
\end{remark}

\subsection{Restricted variance hypothesis}\label{ssec:restricted}

The preceding analysis sums the variance over \emph{all} moduli
$q\le D$, but only the counting factor is restricted to
$\mathcal{Q}_{\mathrm{eff}}$.
A more natural hypothesis restricts \emph{both} factors to the
effective moduli.

\begin{assumption}[Restricted variance hypothesis]
\label{ass:restricted}
There exists a fixed constant $B'>0$ such that
\begin{equation}\label{eq:BDH_restricted}
  \sum_{\substack{q\le D\\
  q\in\mathcal{Q}_{\mathrm{eff}}}}
  E(x,q)^2
  \;\ll\;
  N_{\mathrm{eff}}(D)\cdot x\,(\log x)^{B'},
\end{equation}
i.e.\ the average of $E(x,q)^2$ over effective moduli
$q\le D$ is $\ll x\,(\log x)^{B'}$.
\end{assumption}

\begin{theorem}[Conditional level of distribution]
\label{thm:restricted}
Under Assumption~\textup{\ref{ass:restricted}} with $B'<45/23$,
for any fixed $A>0$ and $D\le x^{1/2}$,
\[
  \sum_{\substack{q\le D\\
  q\in\mathcal{Q}_{\mathrm{eff}}}}
  E(x,q)
  \;\ll_A\;
  x\,(\log x)^{-A}.
\]
In particular, for the standard BDH exponent $B'=1$, we obtain
the level of distribution $\theta=1/2$ for the sequence~$Q(n)$,
with a power saving of $(\log x)^{-11/23}$.
\end{theorem}

\begin{proof}
Applying the Cauchy--Schwarz inequality \emph{within}
$\mathcal{Q}_{\mathrm{eff}}$,
\begin{equation}\label{eq:CS_step2}
  \biggl(
    \sum_{\substack{q\le D\\
    q\in\mathcal{Q}_{\mathrm{eff}}}}
    E(x,q)
  \biggr)^{\!2}
  \;\le\;
  N_{\mathrm{eff}}(D)
  \;\cdot\;
  \sum_{\substack{q\le D\\
  q\in\mathcal{Q}_{\mathrm{eff}}}}
  E(x,q)^2.
\end{equation}
Note that \emph{both} factors on the right now range only over
$\mathcal{Q}_{\mathrm{eff}}$.
Substituting Assumption~\ref{ass:restricted}:
\[
  \text{RHS of \eqref{eq:CS_step2}}
  \;\ll\;
  N_{\mathrm{eff}}(D)
  \cdot N_{\mathrm{eff}}(D)\cdot x\,(\log x)^{B'}
  \;=\;
  \bigl[N_{\mathrm{eff}}(D)\bigr]^2 \cdot x\,(\log x)^{B'}.
\]
By~\eqref{eq:Qeff_count},
$N_{\mathrm{eff}}(D)\asymp D/(\log D)^{45/46}$, so
\[
  \bigl[N_{\mathrm{eff}}(D)\bigr]^2
  \;\asymp\;
  \frac{D^2}{(\log D)^{90/46}}
  \;=\;
  \frac{D^2}{(\log D)^{45/23}}.
\]
For $D\le x^{1/2}$, taking square roots with $D^2\le x$ and
$\log D\asymp\log x$:
\begin{equation}\label{eq:restricted_result}
  \sum_{\substack{q\le D\\
  q\in\mathcal{Q}_{\mathrm{eff}}}}
  E(x,q)
  \;\ll\;
  x\cdot\frac{(\log x)^{B'/2}}{(\log x)^{45/46}}
  \;=\;
  x\,(\log x)^{(23B'-45)/46}.
\end{equation}
The exponent $(23B'-45)/46$ is negative whenever $B'<45/23$.
For $B'=1$, the exponent is $(23-45)/46=-22/46=-11/23<0$,
giving $\sum E\ll x\,(\log x)^{-11/23}$.
Since $11/23>0$, this provides the required logarithmic savings for
$\theta=1/2$.
\end{proof}

\begin{remark}[Double sparsity]\label{rem:double}
The key difference between Proposition~\ref{prop:CS_global} and
Theorem~\ref{thm:restricted} is the \emph{doubling of the sparsity
factor}.
In the global case~\eqref{eq:CS_step1}, the Cauchy--Schwarz step
pairs $N_{\mathrm{eff}}(D)\asymp D/(\log D)^{45/46}$ with a
variance sum over \emph{all} $q\le D$ whose leading factor is~$D$.
After taking the square root, the single factor
$(\log D)^{-45/46}$ contributes $(\log x)^{-45/92}$, which barely
fails to overcome $(\log x)^{B/2}$ when $B=1$.

In the restricted case~\eqref{eq:CS_step2}, both the counting
factor and the variance factor carry $N_{\mathrm{eff}}(D)$,
producing $[N_{\mathrm{eff}}(D)]^2\asymp D^2/(\log D)^{45/23}$.
After the square root, this contributes $(\log x)^{-45/46}$---which
is exactly \emph{twice} the logarithmic saving of the global
case---and comfortably overcomes $(\log x)^{B'/2}$ for any
$B'\le 1$.
\end{remark}

\begin{remark}[Relation to the classical Bombieri--Vinogradov
theorem]\label{rem:BV}
The standard Bombieri--Vinogradov theorem~\cite{Bombieri1965,
Vinogradov1965} already yields $\theta=1/2$ unconditionally for
polynomial sequences.
The contribution of Theorem~\ref{thm:restricted} is not to
reprove this classical result, but to exhibit an independent route
to $\theta=1/2$ in which the algebraic structure of~$Q(n)$ plays
an explicit quantitative role via the double sparsity factor.
\end{remark}

%% ================================================================
\section{Discussion: Toward $\theta>1/2$}\label{sec:beyond}

Theorem~\ref{thm:restricted} shows that the sparsity of
$\mathcal{Q}_{\mathrm{eff}}$ already has quantitative force at the
level $D=x^{1/2}$.
Moving to $D=x^{\theta}$ with $\theta>1/2$ requires methods that
exploit the \emph{multiplicative structure} of
$\mathcal{Q}_{\mathrm{eff}}$, not merely its cardinality.

The Bombieri--Friedlander--Iwaniec method~\cite{BFI1986} achieves
levels of distribution beyond~$1/2$ by decomposing the error sum
into bilinear forms and exploiting cancellation in character sums.
For the sequence $Q(n)$, the null--sparse decomposition ensures
that such bilinear sums need only range over
$q\in\mathcal{Q}_{\mathrm{eff}}$, i.e.\ over moduli composed
entirely of primes $p\equiv 1\pmod{47}$.
This structural restriction reduces both the number of terms and
the range of characters involved, and may facilitate additional
cancellation.

Whether this can be carried out to yield an unconditional level of
distribution $\theta>1/2$ for the sequence $Q(n)$ remains an open
problem.

%% ================================================================
\section{Concluding Remarks}\label{sec:concl}

The cyclotomic norm form $Q(n)=n^{47}-(n-1)^{47}$ possesses a rigid
arithmetic structure that sharply constrains its behavior in
arithmetic progressions.
We have established:
\begin{enumerate}
\item[\textup{(1)}]
The local root structure is completely determined:
$\omega(p)=0$ for $p\not\equiv 1\pmod{47}$ and $p=47$;
$\omega(p)=46$ for $p\equiv 1\pmod{47}$
(Proposition~\ref{prop:local}).
\item[\textup{(2)}]
The set of effective moduli $\mathcal{Q}_{\mathrm{eff}}$ is
sparse, with counting function
$\asymp D(\log D)^{-45/46}$
(Remark~\ref{rem:count}).
\item[\textup{(3)}]
The Bombieri--Vinogradov error sum decomposes into a null part
(supported on non-effective moduli, where the main term vanishes)
and a sparse part (supported on $\mathcal{Q}_{\mathrm{eff}}$)
(Theorem~\ref{thm:decomp}).
\item[\textup{(4)}]
Under a restricted variance hypothesis, the Cauchy--Schwarz
method yields $\theta=1/2$ with a $(\log x)^{-11/23}$ saving,
thanks to the \emph{double sparsity factor}
(Theorem~\ref{thm:restricted}).
\end{enumerate}

No unconditional improvement beyond the classical $\theta=1/2$ is
claimed.
The main contribution is the null--sparse decomposition and the
observation that any improvement in the level of distribution
reduces to a problem on the sparse, algebraically structured set
$\mathcal{Q}_{\mathrm{eff}}$.

\medskip
\noindent\textbf{Companion papers.}
The algebraic, sieve-theoretic, and computational foundations of
$Q(n)$ are developed in~\cite{ChenTitan}.  Large-scale computational
verification ($15.4$ million $Q$-primes for $n\le 2\times10^9$)
and spectral gap analysis appear in~\cite{ChenLS}.

\medskip
\noindent\textbf{Data availability.}
The \LaTeX{} source, verification scripts, and supplementary data
are available at:
\begin{center}
\url{https://github.com/Ruqing1963/Q47-Null-Sparse-Decomposition}
\end{center}

%% ================================================================
\begin{thebibliography}{9}

\bibitem{BFI1986}
E.~Bombieri, J.~Friedlander, and H.~Iwaniec,
\textit{Primes in arithmetic progressions to large moduli},
Acta Math.\ \textbf{156} (1986), 203--251.

\bibitem{BH1962}
P.\,T.~Bateman and R.\,A.~Horn,
\textit{A heuristic asymptotic formula concerning the distribution
of prime numbers},
Math.\ Comp.\ \textbf{16} (1962), 363--367.

\bibitem{Bombieri1965}
E.~Bombieri,
\textit{On the large sieve},
Mathematika \textbf{12} (1965), 201--225.

\bibitem{IK2004}
H.~Iwaniec and E.~Kowalski,
\textit{Analytic Number Theory},
Amer.\ Math.\ Soc.\ Colloq.\ Publ., vol.~53,
AMS, Providence, RI, 2004.

\bibitem{Maynard}
J.~Maynard,
\textit{Small gaps between primes},
Ann.\ of Math.\ (2) \textbf{181} (2015), 383--413.

\bibitem{Vinogradov1965}
A.\,I.~Vinogradov,
\textit{The density hypothesis for Dirichlet $L$-series},
Izv.\ Akad.\ Nauk SSSR Ser.\ Mat.\ \textbf{29} (1965),
903--934.

\bibitem{ChenTitan}
R.~Chen,
\textit{Prime values of a cyclotomic norm polynomial
and a conjectural bounded gap phenomenon},
Preprint (2026),
\url{https://zenodo.org/records/18521551}.

\bibitem{ChenLS}
R.~Chen,
\textit{Experimental constraints on Landau--Siegel zeros: A
2-billion point spectral gap analysis of~$Q_{47}$},
Preprint (2026),
\url{https://zenodo.org/records/18315796}.

\end{thebibliography}

\end{document}
